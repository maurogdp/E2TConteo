\documentclass[11pt, letterpaper]{book}


\usepackage{pgfplots}
\pgfplotsset{compat=1.18}
\usepackage[utf8]{inputenc}
\usepackage[spanish]{babel}
\usepackage{amsmath, amsthm}
\usepackage{amsfonts}
\usepackage{amssymb}
\usepackage{graphicx}
\usepackage[left=2cm,right=2cm,top=2cm,bottom=2cm]{geometry}
\usepackage[export]{adjustbox}
\usepackage{multirow}
\usepackage{multicol}
\usepackage{setspace}
\usepackage{subfig}
\usepackage{venndiagram}
\usepackage{verbatim}
\usepackage{enumitem}
\usepackage{mdframed}
\usepackage{slashbox}


\usepackage{tikz}
\usetikzlibrary{arrows.meta,bbox}
\usetikzlibrary{decorations.pathreplacing}
\begin{comment}
\tikzset{%
  show curve controls/.style={
    postaction={
      decoration={
        show path construction,
        curveto code={
          \draw [blue] 
            (\tikzinputsegmentfirst) -- (\tikzinputsegmentsupporta)
            (\tikzinputsegmentlast) -- (\tikzinputsegmentsupportb);
          \fill [red, opacity=0.5] 
            (\tikzinputsegmentsupporta) circle [radius=.5ex]
            (\tikzinputsegmentsupportb) circle [radius=.5ex];
        }
      },
      decorate
}}}
\end{comment}
\usepackage{stackengine}
\newcommand\xrowht[2][0]{\addstackgap[.5\dimexpr#2\relax]{\vphantom{#1}}}


\usepackage{graphicx}
\usepackage{tikz}
\usetikzlibrary{babel,arrows.meta,decorations.pathmorphing, backgrounds,positioning,fit,petri, shapes, shadows}

\usepackage{tikz,color}
\usepackage{pgf-pie}
\usepackage{pgfplots} 

\theoremstyle{plain}% default
\newtheorem{teo}{Teorema}[section]
\newtheorem{lem}[teo]{Lema}
\newtheorem{prop}[teo]{Proposición}
\newtheorem*{cor}{Corolario}
\newmdtheoremenv{defi}{Definición}[section]
\newtheorem{conj}{Conjetura}[section]
\newtheorem{propi}{Propiedad}[section]

\theoremstyle{definition}
\newtheorem{ejer}{\textit{Ejercicio}}[section]
\newtheorem{ejem}{\textit{Ejemplo}}[section]
\newtheorem*{sol}{Solución}

\theoremstyle{remark}
\newtheorem{obs}{Observación}[section]
\newtheorem*{nota}{Nota}
\newtheorem{caso}{Caso}
\newtheorem*{tips}{Tips}




\newcommand{\paso}[1]{\hspace{1cm}\linebreak\hspace{1cm}\textit{#1 paso. }}


%%%%%%%%%%%%%%%%%%%%%%%%%%%%%%%%%%%%
%%%%%%%%%%%%%%%%%%%%%%%%%%%%%%%%%%%%
%%%%%%%%%%%%%%%%%%%%%%%%%%%%%%%%%%%%

\makeatletter
\newenvironment{myminipage}[1]%
    {\let\@parboxrestore\relax\begin{minipage}{#1}}%
    {\end{minipage}}
\makeatother
%%%%%%%%%%%%%%%%%%%%%%%%%%%%%%%%%%%%%%
%%%%%%%%%%%%%%%%%%%%%%%%%%%%%%%%%%%%%%

\newcounter{conserva}

\newcounter{question}
\newif\ifinchoices
\inchoicesfalse
\newenvironment{questions}{%
  \list{\thequestion.\hspace{0.6cm}}%
  {%
    \usecounter{question}%
    \def\question{\inchoicesfalse\item}%
    \settowidth{\leftmargin}{10.\hskip\labelsep}%
    \labelwidth\leftmargin\advance\labelwidth-\labelsep
  }%
}
{%
  \endlist
}%

\newcounter{choice}
\renewcommand\thechoice{\Alph{choice}}
\newcommand\choicelabel{\thechoice)}
\def\choice{%
  \ifinchoices
    % Do nothing
  \else
    \startchoices
  \fi
  \refstepcounter{choice}%
  %\ifnum\value{choice}>0\relax
  %\penalty -50\hskip 1em plus 1em\relax
  %\fi
  \ifnum\value{choice}>1{\vspace{-0.2cm}}
  
  \fi
  \choicelabel
  \nobreak
  \enskip
}% choice
\def\CorrectChoice{%
  \choice
  \addanswer{\thequestion}{\thechoice}%
}
\let\correctchoice\CorrectChoice

\newcommand{\startchoices}{%
  \inchoicestrue
  \setcounter{choice}{0}%
  \par % Uncomment this to have choices always start a new line
  % If we're continuing the paragraph containing the question,
  % then leave a bit of space before the first choice:
  \ifvmode\else\enskip\fi
}%

\newbox\allanswers
\setbox\allanswers=\hbox{}
\newcommand{\addanswer}[2]{%
  \global\setbox\allanswers=\hbox{\unhbox\allanswers #1.~#2\quad}%
}
\newcommand{\showanswers}{%
  \vfill
  \begin{center}
    Alternativas correctas
  \end{center}
  \noindent\unhbox\allanswers
}

%%%%%%%%%%%%%%%%%%%%%%%%%%%%%%%%%%%%
%%%%%%%%%%%%%%%%%%%%%%%%%%%%%%%%%%%%
%%%%%%%%%%%%%%%%%%%%%%%%%%%%%%%%%%%%










\author{Mauro Díaz}
\title{Apuntes\\Probabilidad y estadística descriptiva e inferencial}

\begin{document}
\pgfplotsset{compat=1.18}
\begin{center}
	\textbf{Evaluación Técnicas de conteo N°2\\Diferenciado de probabilidad y estadística descriptiva e inferencial}
\end{center}

Nombre:$\underline{\hspace{12cm}}$


\begin{questions}


	\question En una fila de 7 sillas se sientan cuatro mujeres y tres hombres, ¿de cuántas maneras se pueden sentar ordenadamente, si las mujeres deben estar juntas y	los hombres también?
	\begin{multicols}{5}
		\choice $2$
		\columnbreak
		\choice $4\cdot 3$
		\columnbreak
		\choice $3!\cdot 4!\cdot 2$
		\columnbreak
		\correctchoice $3!\cdot 4!$
		\columnbreak
		\choice $4\cdot 3\cdot 2$
	\end{multicols}

	\question Si en una tienda de ropa, se deben escoger dos trajes de seis trajes diferentes, ¿de cuántas maneras distintas se puede hacer esta selección?
	\begin{multicols}{5}
		\choice $1$
		\columnbreak
		\choice $15$
		\columnbreak
		\correctchoice $6$
		\columnbreak
		\choice $12$
		\columnbreak
		\choice $3$
	\end{multicols}

	\question Carolina, Daniela, Antonia y Victoria pertenecen a un grupo. Un profesor debe elegir a dos de ellas para realizar un trabajo de matemática. ¿Cuál es el máximo número de combinaciones de parejas que se pueden formar con estas cuatro niñas?
	\begin{multicols}{5}
		\choice $8$
		\columnbreak
		\choice $2$
		\columnbreak
		\correctchoice $6$
		\columnbreak
		\choice $12$
		\columnbreak
		\choice $16$
	\end{multicols}

	\question Se tiene una población compuesta por las fichas 1, 3, 5, 5 y 7. ¿Cuál es la cantidad de todas las posibles muestras (sin reposición y sin orden) de tamaño 2 que pueden extraerse desde esta población?
	\begin{multicols}{5}
		\choice $10$
		\columnbreak
		\choice $20$
		\columnbreak
		\correctchoice $25$
		\columnbreak
		\choice $6$
		\columnbreak
		\choice $12$
	\end{multicols}

	\question Un taller fabrica fichas plásticas y le hacen un pedido de fichas impresas con todos los números de tres dígitos que se pueden formar con el 2, el 3, el 4, el 5 y el 6. ¿Cuál es el doble de la cantidad del pedido?
	\begin{multicols}{5}
		\choice $20$
		\columnbreak
		\choice $30$
		\columnbreak
		\correctchoice $60$
		\columnbreak
		\choice $125$
		\columnbreak
		\choice $250$
	\end{multicols}

	\question El número de todas las posibles muestras distintas, sin orden y sin reposición, de tamaño 3 que se pueden formar con un total de 9 elementos, es
	\begin{multicols}{5}
		\choice $9$
		\columnbreak
		\choice $729$
		\columnbreak
		\correctchoice $27$
		\columnbreak
		\choice $84$
		\columnbreak
		\choice $504$
	\end{multicols}

	\question Un programa computacional genera números de tres dígitos distintos entre sí y ningún dígito puede ser cero. ¿Cuántos de estos números están formados con exactamente dos números primos?
	\begin{multicols}{5}
		\choice $3\binom{4}{2}\binom{5}{1}$
		\columnbreak
		\choice $3\binom{5}{2}\binom{4}{1}$
		\columnbreak
		\correctchoice $6\binom{5}{2}\binom{4}{1}$
		\columnbreak
		\choice $6\binom{4}{2}\binom{5}{1}$
		\columnbreak
		\choice $3\binom{3}{2}\binom{3}{1}$
	\end{multicols}

	\question De un conjunto de n elementos distintos, con $n > 2$, se extraen todas las muestras posibles, sin orden y sin reposición, de tamaño 2. ¿Cuál de las siguientes expresiones representa \textbf{siempre} el número total de estas muestras?
	\begin{multicols}{5}
		\choice $n(n-1)$
		\columnbreak
		\choice $2^n$
		\columnbreak
		\correctchoice $n^2$
		\columnbreak
		\choice $\dfrac{n!}{2!}$
		\columnbreak
		\choice $\binom{n}{2}$
	\end{multicols}
\newpage
	\question ¿Cuántos números distintos divisibles por 2, menores que 100.000 y mayores que 10.000 se pueden formar en total usando los dígitos 3, 4, 5, 7 y 9, considerando que estos se pueden repetir?
	
	\begin{multicols}{5}
		\choice $625$
		\columnbreak
		\choice $20$
		\columnbreak
		\correctchoice $256$
		\columnbreak
		\choice $120$
		\columnbreak
		\choice $24$
	\end{multicols}

	\question De un grupo formado por 5 ingenieros y 6 economistas, todos de distintas edades, se quiere formar una comisión presidida por el ingeniero de más edad del grupo, la cual estará integrada, en total, por 3 ingenieros y 2 economistas. ¿Cuántas comisiones distintas se pueden formar?
	
	\begin{multicols}{5}
		\choice $90$
		\columnbreak
		\choice $210$
		\columnbreak
		\correctchoice $60$
		\columnbreak
		\choice $21$
		\columnbreak
		\choice $360$
	\end{multicols}

	\question ¿Cuántas muestras distintas de tamaño 2 se pueden extraer de una población de 6 elementos distintos entre sí, si las extracciones se hacen sin reemplazo y con orden?
	\begin{multicols}{5}
		\choice $12$
		\columnbreak
		\choice $64$
		\columnbreak
		\correctchoice $30$
		\columnbreak
		\choice $36$
		\columnbreak
		\choice $3$
	\end{multicols}

	\question Se tienen 9 letras diferentes. ¿Cuántas palabras, con o sin sentido, es posible formar con estas 9 letras, sin que se repita ninguna letra, si estas palabras están formadas por al menos 2 letras o a lo más 4 letras?
	\begin{multicols}{5}
		\choice $3!\binom{9}{3}$
		\columnbreak
		\choice $\binom{9}{3}$
		\columnbreak
		\correctchoice $\binom{9}{2}\binom{9}{3}\binom{9}{4}$
		\columnbreak
		\choice $2!3!4!\binom{9}{2}\binom{9}{3}\binom{9}{4}$
		\columnbreak
		\choice $2!\binom{9}{2}+3!\binom{9}{3}+4!\binom{9}{4}$
	\end{multicols}

	\question Para un viaje Andrés arrendará un automóvil en una empresa que le da a elegir entre las marcas P, Q y R. Cada una de estas marcas dispone de dos modelos en los colores blanco, rojo, azul, verde y gris, cada uno de ellos. ¿Cuál es la cantidad máxima de automóviles, de distinto tipo de marca, modelo y color, entre los que Andrés puede elegir?
	\begin{multicols}{5}
		\choice $6$
		\columnbreak
		\choice $10$
		\columnbreak
		\correctchoice $15$
		\columnbreak
		\choice $30$
		\columnbreak
		\choice $90$
	\end{multicols}

	\question ¿Cuántos partidos individuales de tenis se tienen que organizar con n jugadores, donde $n > 2$, si todos juegan contra todos solo una vez?
	\begin{multicols}{5}
		\choice $\dfrac{n}{2}$
		\columnbreak
		\choice $\dfrac{n}{2}\cdot (n-1)$
		\columnbreak
		\correctchoice $n-1$
		\columnbreak
		\choice $n\cdot \dfrac{n}{2}$
		\columnbreak
		\choice $\dfrac{n-1}{2}$
	\end{multicols}

	\question Una persona dispone de 12 lápices de distintos colores para dibujar un adorno que pegará en un muro. Este adorno estará formado por ocho cuadrados en fila, pegados uno al lado del otro. Cada cuadrado debe ser pintado con alguno de los lápices y dos cuadrados seguidos no pueden ser pintados del mismo color.\\
	¿Cuántos adornos distintos con las características antes mencionadas se pueden formar? 
	
	\begin{multicols}{4}
		\choice $8\cdot 12$
		\columnbreak
		\choice $11^8$
		\columnbreak
		\correctchoice $12^8$
		\columnbreak
		\choice $12\cdot 11^7$
	\end{multicols}

	\question ¿Cuántas palabras de 4 letras en total, con sentido y sin él, se pueden formar con las letras de la palabra CUADERNO, si las letras no se pueden repetir? 
	
	\begin{multicols}{5}
		\choice $32$
		\columnbreak
		\choice $8!$
		\columnbreak
		\correctchoice $4!$
		\columnbreak
		\choice $70$
		\columnbreak
		\choice $1680$
	\end{multicols}
\newpage
	\question Jacinta tiene 8 libros de matemática, 7 de literatura y 10 de biología. ¿De cuántas maneras puede escoger 2 libros de cada disciplina para llevarlos al colegio? 
	\begin{multicols}{5}
		\choice $\binom{8}{2}+\binom{7}{2}+\binom{10}{2}$
		\columnbreak
		\choice $\binom{8}{2}\cdot \binom{7}{2}\cdot \binom{10}{2}$
		\columnbreak
		\correctchoice $\binom{25}{2}\cdot \binom{24}{2}\cdot \binom{23}{2}$
		\columnbreak
		\choice $\dfrac{25!}{19!}$
		\columnbreak
		\choice $\dfrac{8!}{2!}\cdot \dfrac{7!}{2!}\cdot \dfrac{10!}{2!}$
	\end{multicols}

	\question En una gira de estudios de un curso, asisten 47 personas entre adultos, niños y niñas, se sabe que son 12 adultos y los niños y niñas están en la razón 3 : 4, respectivamente. Si se quieren formar grupos de 6 personas donde debe haber un adulto, dos niños y tres niñas, ¿cuántos grupos diferentes de personas se pueden formar?
	\begin{multicols}{5}
		\choice $\binom{47}{1}\binom{20}{2}\binom{15}{3}$
		\columnbreak
		\choice $\binom{47}{6}$
		\columnbreak
		\correctchoice $\binom{12}{1}\binom{35}{4}$
		\columnbreak
		\choice $\binom{12}{1}\binom{15}{2}\binom{20}{3}$
		\columnbreak
		\choice $\binom{12}{1}\binom{15}{3}\binom{20}{2}$
	\end{multicols}

	\question Dado el conjunto $A = {1, 2, 3, 4, 5, 6, 7, 8, 9}$, el número de todas las muestras distintas, con orden y sin reposición, de tamaño 6 que se pueden formar, es
	\begin{multicols}{5}
		\choice $\dfrac{9!}{3!}$
		\columnbreak
		\choice $\dfrac{9!}{6!}$
		\columnbreak
		\correctchoice $504$
		\columnbreak
		\choice $84$
		\columnbreak
		\choice $9^6$
	\end{multicols}

	\question ¿Qué valor debe tener n para que se cumpla que $(n - 2)! = \dfrac{1}{2}n!$?
	\begin{multicols}{5}
		\choice $5$
		\columnbreak
		\choice $4$
		\columnbreak
		\correctchoice $3$
		\columnbreak
		\choice $2$
		\columnbreak
		\choice $1$
	\end{multicols}

	\question Un estudiante debe responder 10 de 12 preguntas de un examen, el cual le obliga a responder las 4 primeras, ¿cuántas formas tiene de responder 10 preguntas?
	\begin{multicols}{5}
		\choice $66$
		\columnbreak
		\choice $495$
		\columnbreak
		\correctchoice $28$
		\columnbreak
		\choice $924$
		\columnbreak
		\choice $70$
	\end{multicols}

	\question Roberto todos los días para ir a su trabajo, se cambia de camisa y pantalón. Si Roberto solo dispone de 4 camisas y 3 pantalones, ¿de cuántas maneras puede combinar las dos prendas de vestir?
	\begin{multicols}{5}
		\choice De $7$ maneras.
		\columnbreak
		\choice De $12$ maneras.
		\columnbreak
		\correctchoice De $24$ maneras.
		\columnbreak
		\choice De $3^7$ maneras.
		\columnbreak
		\choice De $7^3$ maneras.
	\end{multicols}

	\question En una caja hay 12 miniaturas que tienen la misma forma, pero 3 de ellas son de madera, 4 son de plástico y 5 son de vidrio. ¿De cuántas maneras diferentes se pueden escoger 3 figuritas de modo que estén fabricadas de materiales diferentes?
	\begin{multicols}{5}
		\choice $12$
		\columnbreak
		\choice $24$
		\columnbreak
		\correctchoice $60$
		\columnbreak
		\choice $150$
		\columnbreak
		\choice Ninguna de las anteriores.
	\end{multicols}

	\question Sean M, N, P, Q, R y S, 6 puntos diferentes de una circunferencia. ¿Cuántas cuerdas diferentes se pueden dibujar uniendo 2 de estos 6 puntos?
	\begin{multicols}{5}
		\choice $6$
		\columnbreak
		\choice $12$
		\columnbreak
		\correctchoice $15$
		\columnbreak
		\choice $30$
		\columnbreak
		\choice $36$
	\end{multicols}
\newpage
	\question Si Arturo está colgando sus 2 pantalones, 4 camisas y 2 chaquetas, de modo que un colgador lleva una sola prenda, ¿de cuántas maneras los puede ordenar dejando al lado izquierdo los pantalones, al centro las camisas y a la derecha las chaquetas?
	\begin{multicols}{5}
		\choice $4$
		\columnbreak
		\choice $8$
		\columnbreak
		\correctchoice $16$
		\columnbreak
		\choice $48$
		\columnbreak
		\choice $96$
	\end{multicols}

	\question Utilizando solo los números de 1 a 9, el total de números posibles de 4 dígitos, que sean múltiplos de 5, sin repetición, son
	\begin{multicols}{5}
		\choice $\dfrac{8!}{(8-3)!}$
		\columnbreak
		\choice $\dfrac{8!}{3!}$
		\columnbreak
		\correctchoice $\dfrac{8!}{4!}$
		\columnbreak
		\choice $\binom{8}{3}$
		\columnbreak
		\choice $\binom{8}{4}$
	\end{multicols}

	\question En un club de campo se realiza un campeonato de dobles en tenis cuyas parejas participantes provienen de dos grupos. En el grupo A hay 6 jugadores y en el grupo B hay 4 jugadores. ¿Cuántos partidos podrían jugarse en el campeonato si siempre una pareja debe pertenecer al grupo A y la otra pareja al grupo B?
	\begin{multicols}{5}
		\choice $10$
		\columnbreak
		\choice $24$
		\columnbreak
		\correctchoice $90$
		\columnbreak
		\choice $180$
		\columnbreak
		\choice $240$
	\end{multicols}

	\question Si de una población se pueden extraer 45 muestras de 2 elementos, sin orden ni reemplazo. Entonces, el tamaño de la población es de
	\begin{multicols}{5}
		\choice $40$
		\columnbreak
		\choice $15$
		\columnbreak
		\correctchoice $17$
		\columnbreak
		\choice $10$
		\columnbreak
		\choice $9$
	\end{multicols}

	\question Para un trabajo didáctico, un niño debe elegir tres láminas de goma Eva de colores distintos, de una gama de 8 colores disponibles, para rellenar las tres regiones de un círculo y además dos láminas de colores diferentes de papel lustre, de entre 5 colores que se le ofrecen, para rellenar las dos mitades de un cuadrado. ¿De cuántas formas distintas el niño puede hacer el trabajo?
	\begin{multicols}{5}
		\choice $2\cdot \binom{8}{3}\cdot \binom{5}{2}$
		\columnbreak
		\choice $\dfrac{8!}{3!}\cdot \dfrac{5!}{2!}$
		\columnbreak
		\correctchoice $2\cdot \dfrac{8!}{3!}\cdot \dfrac{5!}{2!}$
		\columnbreak
		\choice $8!\cdot 3!\cdot 5!\cdot 2!$
		\columnbreak
		\choice $\binom{8}{3}\cdot \binom{5}{2}$
	\end{multicols}

	\question En una audición de teatro hay 20 hombres y 5 mujeres. ¿De cuántas maneras pueden elegirse 3 hombres y una mujer para formar el elenco?
	\begin{multicols}{5}
		\choice $20\cdot 19\cdot 18\cdot 5$
		\columnbreak
		\choice $\binom{25}{4}$
		\columnbreak
		\correctchoice $\binom{20}{3}\binom{5}{1}$
		\columnbreak
		\choice $\binom{20}{3}+5$
		\columnbreak
		\choice $\dfrac{3}{20}+\dfrac{1}{5}$
	\end{multicols}

\end{questions}

\end{document}



